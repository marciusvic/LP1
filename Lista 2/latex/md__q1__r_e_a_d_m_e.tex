\#\+Questão 1 b)Recursão de cauda é um tipo de recursão em que uma função chama a si mesma como sua última operação, sem precisar realizar outras operações depois da chamada recursiva. No cógido ela causa um menor uso de memória e maior eficiência, na recursão de cauda a chamada recursiva é a última operação a ser executada, permitindo que a mesma pilha de execução seja utilizada. A solução é transformar chamadas recursivas em loops, evitando a alocação de novos espaços na pilha de execução. c)Não, na função tribonacci\+\_\+recursiva a chamada recursiva não é a última operação a ser executada antes do retorno da função, as chamadas recursivas são realizadas, mas em seguida há uma operação de soma e só então tudo é retornado. Portanto, o retorno das chamadas recursivas não é a última operação a ser executada. d)O Stack Overflow ocorre quando uma função recursiva ou um programa em geral começa a usar mais espaço de memória do que o computador tem disponível. Quando isso acontece, o programa tenta armazenar informações na memória que já está sendo usada, o que faz com que ocorra um erro. Isso pode acontecer com funções recursivas porque cada vez que a função é chamada, uma nova \char`\"{}pilha\char`\"{} de informações é adicionada na memória. Se a função for chamada muitas vezes, a pilha pode ficar muito grande e exceder o espaço de memória disponível, o que pode levar ao Stack Overflow. 